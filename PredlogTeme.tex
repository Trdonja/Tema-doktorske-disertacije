%----------------------------------------------------------------------------------------------------
% (c) 2017 Domen Močnik & Tomaž Vrtovec
% Predlog teme doktorske disertacije
% [verzija v slovenskem jeziku]
%----------------------------------------------------------------------------------------------------


%-------------------------
% define document style: A4 paper size, two-sided printing, 11pt font, article style
\documentclass[a4paper,twoside,11pt]{article}
%-------------------------


%-------------------------
% use the follwing packages
\usepackage[slovene]{babel}
\usepackage[cp1250]{inputenc}
\usepackage[pdftex]{graphicx}
\usepackage{verbatim}
\usepackage{eso-pic}
\usepackage{color}
\usepackage{amsmath}
\usepackage{amssymb}
\usepackage{enumitem}
\usepackage{amsmath}
\usepackage{amssymb}
\usepackage{dsfont}
%-------------------------


%-------------------------
% spacing between lines
%\renewcommand{\baselinestretch}{1.5}
% side margin on odd pages
\oddsidemargin 0.2in
% side margin on even pages
\evensidemargin -0.1in
% for top margin of 1.0in
\topmargin 0.0in
% width  of text area (=  8.5in - 2*1.1in)
\textwidth  6.3in
% height of text area (= 11.0in - 2*1.0in)
\textheight 9.0in
% avoid extra space for header
\headheight 0.7in \headsep 0.5in
\voffset -1in
% baseline-baseline distance between running footline and last line of text
\footskip 1in
% no extra vertical space is added to pages
\raggedbottom

% put "." behind section numbers
\renewcommand \thesection{\arabic{section}.}
\renewcommand \thesubsection{\arabic{section}.\arabic{subsection}.}

% fancy headers and footers (package: fancyhdr)
\usepackage{fancyhdr}
\pagestyle{fancy}
%\renewcommand{\sectionmark}[1]{\markboth{\thesection\ #1}{}}
\fancyhf{}
\fancyfoot[CO,CE]{\thepage}
%\fancyhead[LE,RO]{\slshape \leftmark}
%\fancyhead[LE,RO]{\slshape Prijava teme doktorske disertacije}
%\renewcommand{\headrulewidth}{0.4 pt}
\renewcommand{\headrulewidth}{0.0 pt}
\renewcommand{\footrulewidth}{0.0 pt}

% bibliography (package: natbib)
\usepackage[round,comma,sort&compress]{natbib}
\setlength{\bibsep}{0.15cm} % set the bibliography line spacing

% hyperlinks
\usepackage[unicode=true,hyperfootnotes,hyperindex]{hyperref}
\hypersetup {
	colorlinks=false,
	linkcolor=red,
	pdftitle={Prijava teme doktorske disertacije},
	pdfauthor={Domen Močnik},
	pdfsubject={Slikovno vodena radioterapija glave in vratu},
	pdfkeywords={}
}

% footnotes
\usepackage{fnpos}
%\usepackage{dblfnote}
%\DFNcolumnsep 1.5\columnsep % set the distance between footnote columns
\makeatletter % set space between text and footnote (1em = one line)
\renewcommand\footnoterule{
	\vspace{1.5em}
	\kern-3\p@\hrule\@width.4\columnwidth
	\kern2.6\p@}
\makeatother

% table of contents (package: tocloft)
\usepackage{tocloft}
\renewcommand{\cftsecleader}{\bfseries\cftdotfill{\cftsubsecdotsep}} % define that section title is followed by bold dots to the number
\setlength{\cftaftertoctitleskip}{2.0cm} % space after toc title
\setlength{\cftbeforesecskip}{1.0cm} % space between sections
\setlength{\cftbeforesubsecskip}{0.5cm} % space between subsections
%-------------------------


%-------------------------
% BEGIN DOCUMENT
\begin{document}
	%-------------------------
	% TITLE PAGE
	\thispagestyle{empty}
	\par{ \noindent
		\begin{center}
			\includegraphics[width=5cm]{../images/fe_logo_SLO} \\ \vspace{2.0cm}
			\textbf{\Large Prijava teme doktorske disertacije} \\ \vspace{2.0cm}
			\textsc{\Huge Slikovno vodena} \\ \vspace{0.3cm}
			\textsc{\Huge radioterapija} \\ \vspace{0.5cm}
			\textsc{\Huge glave in vratu} \\ \vspace{0.5cm}
			\textsc{\Huge še naslov} \\ \vspace{1.0cm}
			\textsc{\Huge Image guided} \\ \vspace{0.5cm}
			\textsc{\Huge radiotherapy} \\ \vspace{0.5cm}
			\textsc{\Huge of head and neck} \\ \vspace{0.5cm}
			\textsc{\Huge more title} \\ \vspace{2.2cm}
			\textbf{\Large Domen Močnik} \\ \vspace{2.0cm}
			\textbf{\large Ljubljana, 21. marec 2017 }
		\end{center}
	}
	\newpage
	% empty page after title page
	\thispagestyle{empty}
	~\\
	\newpage
	%-------------------------
	
	
	%-------------------------
	% TABLE OF CONTENTS
	\thispagestyle{empty}
	% change table of contents title
	\renewcommand\contentsname{Vsebina}
	\tableofcontents
	% empty page after table of contents
	\newpage
	\thispagestyle{empty}
	~\\
	\newpage
	%-------------------------
	
	
	%-------------------------
	% PREDLOG TEME DOKTORSKE DISERTACIJE
	\section{Predlog teme doktorske disertacije}
	%-------------------------
	% OPIS OZJEGA ZNANSTVENEGA PODROCJA IN PROBLEMATIKE
	\subsection{Opis ožjega znanstvenega področja in problematike}
	%-------------------------
	% UVOD
	\subsubsection*{Uvod}
	\par{}
	%-------------------------
	% NASLEDNJI RAZDELEK
	\subsubsection*{}
	\par{}
	%-------------------------
	% PRICAKOVANI IZVIRNI PRISPEVKI K ZNANOSTI
	\subsection{Pričakovani izvirni prispevki k znanosti}
	\par{\noindent
		Pričakovani izvirni prispevki k znanosti v predlagani doktorski disertaciji so naslednji:
		\begin{enumerate}
			% 1. IZVIRNI PRISPEVEK
			\item Vrednotenje in primerjava različnih algoritmov za poravnavo slik na bazi slik glave in vratu
			% 2. IZVIRNI PRISPEVEK
			\item Metoda za razgradnjo večmodalnih slik
			% 3. IZVIRNI PRISPEVEK
			\item Izmisli si še kaj.

		\end{enumerate}
	}
	%-------------------------
	% METODOLOGIJA
	\subsection{Metodologija}
	\par{\noindent
		V skladu s predlaganimi pričakovanimi izvirnimi prispevki k znanosti je v nadaljevanju podana ustrezna metodologija:
		\begin{enumerate}
			% 1. IZVIRNI PRISPEVEK
			\item En odstavek.
			% 2. IZVIRNI PRISPEVEK
			\item En odstavek.
			% 3. IZVIRNI PRISPEVEK
			\item En odstavek.
			
		\end{enumerate}
	}
	%-------------------------
	\renewcommand{\bibsection}{\subsection{Izbrana literatura}}
	\bibliographystyle{./stil_literature.bst} % select author-year bibliography style in Slovenian language
	\bibliography{./literatura.tex}
	\newpage
	%-------------------------
	% ŽIVLJENJEPIS
	\newpage
	\section{Življenjepis}
	
	% Osebni podatki
	\subsection*{Osebni podatki}
	\begin{itemize}[align=left, itemsep=-0.1cm, leftmargin=4.25cm, labelwidth=*]
		\item[Ime in priimek:] Domen Močnik
		\item[Datum in kraj rojstva:] 24.11.1986, Novo mesto
		\item[Stalno bivališče:] Stopiče 73, 8322 Stopiče
		\item[Državljanstvo:] slovensko
	\end{itemize}
	
	% Kratek življenjepis
	\subsection*{Kratek življenjepis}
	\par{
		Domen Močnik se je rodil 24. novembra 1986 v Novem mestu. Po opravljeni poklicni maturi na Ekonomski šoli Novo mesto se je leta 2005 vpisal na visokošolski študijski program Praktična matematika na Fakulteti za matematiko in fiziko, Univerza v Ljubljani, na katerem je leta 2010 diplomiral z zagovorom diplomske naloge \textit{Večplastni piezoelektrični nosilec}. Študij je na isti fakulteti nadaljeval z vpisom v 3. letnik prvostopenjskega univerzitetnega študija Matematika, po zaključku pa se je leta 2012 na isti fakulteti vpisal na II. bolonjsko stopnjo programa Matematika. Septembra 2015 je zaključil študij z zagovorom magistrske naloge \textit{Hamiltonov princip v mehaniki kontinuuma}. Septembra 2015 se je vpisal na doktorski študij elektrotehnike na Fakulteti za elektrotehniko, Univerza v Ljubljani, ter se zaposlil kot mladi raziskovalec v Laboratoriju za slikovne tehnologije, kjer trenutno raziskuje na področju avtomatske računalniško podprte obdelave in analize medicinskih slik.
	}
	
	% Izobrazba
	\subsection*{Izobrazba}
	\begin{itemize}[align=left, itemsep=-0.1cm, leftmargin=1cm, labelwidth=*]
		\item[2005] Opravil poklicno maturo na Ekonomski šoli Novo mesto \\ Naziv: ekonomski tehnik
		\item[2010] Diplomiral na visokošolskem programu Praktična matematika na Fakulteti za matematiko in fiziko, Univerza v Ljubljani \\ Naziv: diplomirani inženir matematike
		\item[2012] Diplomiral na univerzitetnem programu Matematika na Fakulteti za matematiko in fiziko, Univerza v Ljubljani \\ Naziv: diplomirani matematik
		\item[2015] Magistriral na Fakulteti za matematiko in fiziko, Univerza v Ljubljani \\ Naziv: magister matematike
		\item[2015] Vpis na doktorski študij na Fakulteti za elektrotehniko, Univerza v Ljubljani \\ študijska smer: elektrotehnika
	\end{itemize}
	
	% Programi
	\subsection*{Programi}
	\begin{itemize}[align=left, itemsep=-0.1cm, leftmargin=2.5cm, labelwidth=*]
		\item[2014~-~\textit{danes}] P2-0232: Funkcije in tehnologije kompleksnih sistemov (Fakulteta za elektrotehniko, Univerza v Ljubljani)
	\end{itemize}
	

	%-------------------------
	
	
	%-------------------------
	% BIBLIOGRAFIJA
	\newpage
	\section{Bibliografija}
	\par{\noindent
		\textbf{Osebna bibliografija za obdobje 2010--2017\footnote{vir bibliografskih zapisov: vzajemna baza podatkov COBISS.SI/COBIB.SI (\href{http://www.cobiss.si}{http://www.cobiss.si}) z dne 25.5.2016}: \\}
		\par{\noindent \textbf{Monografije in druga zaključena dela} \\}
		\vspace{-0.3cm}
		\par{\noindent \textbf{[2.11] Diplomsko delo}}
		\begin{itemize}[align=right, itemsep=-0.05cm]
			\item[1.] MOČNIK, Domen. Večplastni piezoelektrični nosilec : diplomska naloga. Ljubljana, 2010. [COBISS.SI-ID 15732057]
			\item[2.] MOČNIK, Domen. Zgornja polravnina kot model za hiperbolično geometrijo : delo diplomskega seminarja. Ljubljana, 2012. [COBISS.SI-ID 16659289]
		\end{itemize}
		\par{\noindent \textbf{[2.9] Magistrsko delo}}
		\begin{itemize}[align=right, itemsep=-0.05cm]
			\item[3.] MOČNIK, Domen. Hamiltonov princip v mehaniki kontinuuma : magistrsko delo. Ljubljana, 2015. [COBISS.SI-ID 17443673]
		\end{itemize}

		%-------------------------
		\newpage
		\section{Izjava o sekundarnih podatkih}
		\par{		
		Prijavitelj teme doktorske disertacije Domen Močnik ter predvideni mentor doktorske disertacije izr.\ prof.\ dr.\ Tomaž Vrtovec izjavljava, da raziskave v povezavi z izdelavo doktorske disertacije z naslovom ``Slikovno vodena radioterapija glave in vratu'' temeljijo na sekundarnih podatkih in da so bili ti podatki v okviru raziskav anonimizirani.
		}
		\\\\\\
		\noindent\makebox[3cm][l]{Datum: 21.3.2017}
		\\\\\\
		\noindent\makebox[3cm][l]{Prijavitelj teme:} \hfill\makebox[5cm][l]{Predvideni mentor:}
		\\\\\\
		\noindent\makebox[3cm][l]{Domen Močnik, mag.\ mat.} \hfill\makebox[5cm][l]{izr.\ prof.\ dr.\ Tomaž Vrtovec}\\
		
		%-------------------------
		% DIPLOMA
		\newpage
		\section{Potrdilo o končanem izobraževanju}
		\par{ \noindent
			~\\
			\begin{center}
				\fbox{\includegraphics[angle=90,origin=c,width=14.25cm]{./fotokopija_diplome.pdf}}
			\end{center}
		}
		%-------------------------
		
		%-------------------------
		% END DOCUMENT
	\end{document}
	%-------------------------
	