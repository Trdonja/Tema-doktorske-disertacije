%----------------------------------------------------------------------------------------------------
% (c) 2017 Domen Močnik & Tomaž Vrtovec
% Predlog teme doktorske disertacije
% [verzija v slovenskem jeziku]
%----------------------------------------------------------------------------------------------------


%-------------------------
% define document style: A4 paper size, two-sided printing, 11pt font, article style
\documentclass[a4paper,twoside,11pt]{article}
%-------------------------


%-------------------------
% use the follwing packages
\usepackage[slovene]{babel}
%\usepackage[cp1250]{inputenc}
\usepackage[utf8]{inputenc}
\usepackage[pdftex]{graphicx}
\usepackage{verbatim}
\usepackage{eso-pic}
\usepackage{color}
\usepackage{amsmath}
\usepackage{amssymb}
\usepackage{enumitem}
\usepackage{amsmath}
\usepackage{amssymb}
\usepackage{dsfont}
%-------------------------


%-------------------------
% spacing between lines
%\renewcommand{\baselinestretch}{1.5}
% side margin on odd pages
\oddsidemargin 0.2in
% side margin on even pages
\evensidemargin -0.1in
% for top margin of 1.0in
\topmargin 0.0in
% width  of text area (=  8.5in - 2*1.1in)
\textwidth  6.3in
% height of text area (= 11.0in - 2*1.0in)
\textheight 9.0in
% avoid extra space for header
\headheight 0.7in \headsep 0.5in
\voffset -1in
% baseline-baseline distance between running footline and last line of text
\footskip 1in
% no extra vertical space is added to pages
\raggedbottom

% put "." behind section numbers
\renewcommand \thesection{\arabic{section}.}
\renewcommand \thesubsection{\arabic{section}.\arabic{subsection}.}

% fancy headers and footers (package: fancyhdr)
\usepackage{fancyhdr}
\pagestyle{fancy}
%\renewcommand{\sectionmark}[1]{\markboth{\thesection\ #1}{}}
\fancyhf{}
\fancyfoot[CO,CE]{\thepage}
%\fancyhead[LE,RO]{\slshape \leftmark}
%\fancyhead[LE,RO]{\slshape Prijava teme doktorske disertacije}
%\renewcommand{\headrulewidth}{0.4 pt}
\renewcommand{\headrulewidth}{0.0 pt}
\renewcommand{\footrulewidth}{0.0 pt}

% bibliography (package: natbib)
\usepackage[round,comma,sort&compress]{natbib}
\setlength{\bibsep}{0.15cm} % set the bibliography line spacing

% hyperlinks
\usepackage[unicode=true,hyperfootnotes,hyperindex]{hyperref}
\hypersetup {
	colorlinks=false,
	linkcolor=red,
	pdftitle={Prijava teme doktorske disertacije},
	pdfauthor={Domen Močnik},
	pdfsubject={Slikovno vodena radioterapija glave in vratu},
	pdfkeywords={}
}

% footnotes
\usepackage{fnpos}
%\usepackage{dblfnote}
%\DFNcolumnsep 1.5\columnsep % set the distance between footnote columns
\makeatletter % set space between text and footnote (1em = one line)
\renewcommand\footnoterule{
	\vspace{1.5em}
	\kern-3\p@\hrule\@width.4\columnwidth
	\kern2.6\p@}
\makeatother

% table of contents (package: tocloft)
\usepackage{tocloft}
\renewcommand{\cftsecleader}{\bfseries\cftdotfill{\cftsubsecdotsep}} % define that section title is followed by bold dots to the number
\setlength{\cftaftertoctitleskip}{2.0cm} % space after toc title
\setlength{\cftbeforesecskip}{1.0cm} % space between sections
\setlength{\cftbeforesubsecskip}{0.5cm} % space between subsections
%-------------------------


%-------------------------
% BEGIN DOCUMENT
\begin{document}
	%-------------------------
	% TITLE PAGE
	\thispagestyle{empty}
	\par{ \noindent
		\begin{center}
			\includegraphics[width=5cm]{./fe_logo_SLO} \\ \vspace{2.0cm}
			\textbf{\Large Prijava teme doktorske disertacije} \\ \vspace{2.0cm}
			\textsc{\Huge Slikovno vodena} \\ \vspace{0.5cm}
			\textsc{\Huge radioterapija} \\ \vspace{0.5cm}
			\textsc{\Huge glave in vratu} \\ \vspace{0.5cm}
			\textsc{\Huge še naslov} \\ \vspace{1.0cm}
			\textsc{\Huge Image guided} \\ \vspace{0.5cm}
			\textsc{\Huge radiotherapy} \\ \vspace{0.5cm}
			\textsc{\Huge of head and neck} \\ \vspace{0.5cm}
			\textsc{\Huge more title} \\ \vspace{2.2cm}
			\textbf{\Large Domen Močnik} \\ \vspace{2.0cm}
			\textbf{\large Ljubljana, 21. marec 2017 }
		\end{center}
	}
	\newpage
	% empty page after title page
	\thispagestyle{empty}
	~\\
	\newpage
	%-------------------------
	
	
	%-------------------------
	% TABLE OF CONTENTS
	\thispagestyle{empty}
	% change table of contents title
	\renewcommand\contentsname{Vsebina}
	\tableofcontents
	% empty page after table of contents
	\newpage
	\thispagestyle{empty}
	~\\
	\newpage
	%-------------------------
	
	
	%-------------------------
	% PREDLOG TEME DOKTORSKE DISERTACIJE
	\section{Predlog teme doktorske disertacije}
	%-------------------------
	% OPIS OZJEGA ZNANSTVENEGA PODROCJA IN PROBLEMATIKE
	\subsection{Opis ožjega znanstvenega področja in problematike}
	%-------------------------
	% UVOD
	\subsubsection*{Uvod}
	\par{
	  Radioterapija je poleg kirurških posegov in kemoterapije najbolj pomembna metoda zdravljenja raka. Uporablja se pri zdravljenju več kot polovice rakavih bolnikov bodisi kot samostojna metoda bodisi v kombinaciji s katero drugo metodo. Pred kirurškimi posegi se z radioterapijo poskuša zmanjšati velikost tumorjev, s čimer se posege prostorsko omeji, po kirurških posegih pa se z radioterapijo uničuje mikroskopske tumorske celice, ki še ostanejo po kirurških odstranitvah tumorjev. Kadar odstranitev tumorja ni mogoča, se radioterapija uporablja kot paliativno zdravljenje za lajšanje bolečin, pritiska in ostalih simptomov, s tem ko krči tumor. Zdravljenje raka z radioterapijo postaja z razvojem tehnologij vse bolj učinkovito in cenovno dostopno. Slaba stran radioterapije je ta, da se pri obsevanju tumorja nenamerno povzroča tudi poškodbe okoliškega zdravega tkiva, ki je pomembno za zdravo delovanje pacientovega organizma, zaradi česar se povzroča razne neželjene stranske učinke. Tovrstnim poškodbam se je v določeni meri mogoče izogniti z izboljšano natančnostjo dostavljanja sevalne doze, tako da jo čim bolj omejimo na območje tumorja. Izboljševanje natančnosti obsevanja tako predstavlja največji izziv pri izboljševanju zdravljenja z radioterapijo. K temu so v preteklih desetletjih precej pripomogle naprave za natančnejše oblikovanje sevalnih polj, računalniško podprto načrtovanje obsevanja ter slikovne tehnologije, ki omogočajo notranji vpogled v anatomijo pacienta ter lokalizacijo tumorja ter kritičnih okoliških struktur.
	} %TODO: Zakaj sta pa pomembni razgradnja in poravnava slik? Tega nič ne omenjaš, pa ravno s tem se boš ukvarjal.
	\subsubsection*{Delovanje radioterapije}
	\par{
	  Namen radioterapije je uničenje rakavih celic s pomočjo ionizirajočega sevanja. Izsevani električno nabiti delci (ioni) na poti skozi biološko tkivo sproščajo svojo energijo, zadosti sproščene enerigje pa neposredno ali pa posredno poškoduje genetski material celic (DNK), s čimer celice izgubijo zmožnost za nadaljnjo delitev in sčasoma odmrejo. Glede na način dostavljanja sevanja v celice se radioterapija deli na \emph{teleradioterapijo} ali \emph{zunanje obsevanje}, kjer se žarki sevanja oblikujejo zunaj človeškega telesa in se jih usmerja v telo bolnika, ter \emph{brahiterapijo} ali \emph{notranje obsevanje}, kjer se izvor sevanja, t.~j. radioaktivno snov, vsadi v notranost telesa. Glede na izvor se zunanja sevanja v grobem delijo v dve skupini. V prvo skupino spadajo elektromagnetna sevanja in sicer X-žarki ter $\gamma$-žarki. Ionizirajoči delci v tovrstnih žarkih so brezmasni fotoni, zato se tovrstno sevanje imenuje tudi fotonsko sevanje. Za proizvajanje X-žarkov se uporabljajo katodne cevi in linearni pospeševalniki, $\gamma$-žarki pa se proizvajajo pri razpadu radioaktivnih snovi. Fotonski žarki prenašajo nizko energijo in so razpršeni, zato se jih v sodobni radioterapiji nadomešča z drugo skupino sevanja, to so sevanja delcev (elektronov, protonov in nevtronov) ali težjih ionov (ogljik), ki so zaradi višje energije in manj razpršenih žarkov bolj učinkovita in natančna. Tovrstna sevanja se proizvajajo v sinhrotronih, ciklotronih in generatorjih nevtronov in so dražja. \cite{baskar2012}
	}
	\par{
	  Delci sevanja svojo energijo prenašajo na snov, skozi katero prehajajo. Doza sevanja je količina nakopičene izsevane energije v tkivu. Učinek uničevanja celic s sevanjem je večji, če je večja energija sevanja ter če je tkivo izpostavljeno sevanju dlje časa, odvisen pa je tudi od občutljivosti tkiva na sevanje. Pri obsevanju tumorja je neizogibno, da se del energije prenese tudi na zdrave celice v okolici tumorja in na poti do njega, zato so te celice prav tako prizadete. Cilj radioterapije je dostaviti čim večjo dozo sevanja v rakave celice in hkrati čim manjšo dozo v okoliško zdravo tkivo. Visoko energijski protoni ali težji ioni dosežejo svojo maksimalno energijsko izgubo na enoto dolžine tik pred koncem svojega dosega. V okolici tega maksimuma, ki se imenuje \emph{Braggov vrh}, se sprosti največ energije, njegovo lokacijo pa je mogoče prilagoditi glede na lokacijo tumorja, tako da se največ energije sprosti v rakave celice in čim manj v okoliško tkivo. To daje protonski radioterapiji veliko prednost \cite{gregoire2015}.
	}
	\subsubsection*{Intenzitetno modulirana radioterapija}
	\par{
	  Protonski žarek, ki prihaja iz pospeševalnika, ima majhen premer in ozek Braggov vrh (večina energije se sprosti v zelo majhni okolici Braggovega vrha), zato je potrebno žarek bodisi razpršiti (razpršitvena tehnika), tako da prečni presek žarka pokriva prečni presek tumorja vzdolž žarka, bodisi je potrebno žarek preusmerjati v različne predele tumorja in tako ``poslikati dozo'' (skeniranje z ozkim žarkom). Pri razpršitveni tehniki (angl. \emph{scattering technique}) se žarek razpši s kosom materiala, ki žarke ukloni. Nato je potrebno časovno spreminjati lokacijo Braggovega vrha, tako da se sproščana energija homogenizira po celotni globini tumorja. To se doseže tako, da se žarek pošlje skozi plast vode v kolesu, katerega os vrtenja je vzporedna smeri žarka in katerega debelina se spreminja po obsegu na polmeru, kjer žarek prebada površino kolesa. Debelina plasti vode, ki jo žarek preči, določa končni doseg žarka, z vrtenjem kolesa pa se modulira doseg žarka po celotni globini tumorja. Obliko prečnega preseka razpršenega žarka določa kolimator, postavljen pred pacienta. Kolimator je sestavljen iz kovinskih lističev, s pomočjo katerih se da sestaviti binarno masko: delci, ki se zaletijo v kolimator, ne dosežejo bolnika, ostali potujejo dalje neovirano. Včasih se dodatno modulira doseg žarkov še z voščeno snovjo (angl. \emph{bolus}), ki se jo postavi na pacientovo kožo.
	}
	\par{
	  Pri skeniranju z ozkim žarkom (angl. \emph{pencil beam scanning}) je možno vpadno smer žarka hitro spremeninjati z magnetnim poljem, ki se ga ustvari z magneti, med katerimi žarek potuje. Doseg žarka se modulira z nastavljanjem energije žarka, preden ta doseže izstrelitveno rampo pospeševalnika (angl. \emph{gantry}). Izstrelitveno rampo je v nekaterih primerih mogoče vrteti okoli bolnika, s čimer se da dodatno spreminjati vstopno smer žarkov. S tovrstnim obsevanjem se v časovnih intervalih zaporedoma obseva več različnih majhnih področij tumorja, ali pa se žarek zvezno pomika po trajektoriji v prečni ravnini, kjer se intenziteta žarka spreminja po časovno odvisnem vzorcu za trajektorijo. Razpršitvena tehnika je preprostejša in zaenkrat bolj razširjena v klinični uporabi, skeniranje z ozkim žarkom pa omogoča natančnejšo porazdelitev doze, hkrati pa je hitrejše (ne potrebuje prenastavljanje kolimatorja, bolusa in kolesa) in vsi delci dosežejo cilj (ni izgube na kolimatorju). \cite{schippers2011}
	}
	\par{
	  Ker nam razpršitvena tehnika, še bolj pa skeniranje z ozkim žarkom, z nastavitvijo opisanih instrumentov omogočata oblikovanje zapletenih tridimenzionalnih (3D) tarčnih oblik in tako natančno porazdelitev sevalne doze, ki se dobro prilega tumorju, je pomembno natančno poznavanje lege in oblike tumorja in okoliških kritičnih struktur. V ta namen se zajeme načrtovalna 3D slike pacientovega področja, kjer se nahaja tumor, na njej pa zdravnik oriše področje tumorja in področja okoliških kritičnih struktur. Na podlagi orisanih struktur računalniški program izračuna optimalno porazdelitev sevalne doze v okviru možnih nastavitev instrumentov (\emph{inverzno načrtovanje}). Obsevanje je na podlagi tega izračuna računalniško vodeno. Celotni opisani postopek se imenuje \emph{intenzitetno modulirana radioterapija}. \cite{jaffray2012}
	}
	\subsubsection*{Slikovno vodena radioterapija}
	\par{
	  Za načrtovanje radioterapije se zajeme 3D CT (angl. \emph{computed tomography}) sliko ali magnetno resonančno (MR) sliko. Obe vrsti slik prikazujeta notranjo strukturo in anatomsko zgradbo pacienta, pri čemer so na CT slikah bolje vidne kostne strukture in strukture z višjo elektronsko gostoto, MR slike pa ponujajo boljši kontrast na področju mehkih tkiv. CT slike so geometrijsko natančne (brez distorzije), kar za MR slike ne velja. V primerih, kjer sta za načrtovanje na voljo obe vrsti slik, lahko z dobro geometrijsko poravnavo slik združimo informacije, ki jih ponujata obe sliki. S poravnavo MR slike na CT sliko dobimo tudi geometrijsko nepopačeno MR sliko.
	}
	\par{
	  Zdravljenje z radioterapijo običajno poteka v več delih (frakcijah) z dnevnimi presledki skozi večtedensko obdobje. V tem obdobju lahko pride do večjih volumskih sprememb anatomskih struktur v obsevanem območju, npr. pri obsevanju predela glave in vratu se krči tumor \cite{surucu2016} in parotidne žleze \cite{fiorentino2012}, pacient izbublja telesno maso \cite{ottosson2013}, določena tkiva lahko zatečejo. Nadalje med posameznimi frakcijami lahko prihaja do odstopanj v legi pacienta. Zato sčasoma načrtovalna slika ne odraža več natančnega položaja orisanih območij. Načrtovana porazdelitev sevalne doze tako lahko ne zajema več celotnega tumorja, hkrati pa lahko sevanju bolj izpostavi okoliško zdravo ali kritično tkivo. Sodobne naprave za radioterapijo združujejo tudi CT ali pa CBCT (CT s stožčastim žarkom, angl.~cone beam CT) skenerje, s katerimi je mogoče zajeti sliko pacienta v položaju za obsevanje nekaj trenutkov pred začetkom frakcije (medterapevtska slika), ki natančno odraža aktualni položaj in se lahko uporabi za posodobitev načrtovane porazdelitve sevalne doze. Ročno orisovanje volumnov tumorja in kritičnih struktur je zamudno opravilo, zato se orisovanje opravi zgolj enkrat, na načrtovalni sliki, nato pa se pred vsako frakcijo z avtomatskimi postopki poravna načrtovalno sliko na aktualno medterapevtsko sliko, s čimer se poravna tudi orisane volumne. Za tem se ponovno izvede inverzno načrtovanje. Opisani postopek se imenuje slikovno vodena radioterapija. Ker je torej natančnost slikovno vodene radioterapije v veliki meri odvisna od natančnosti geometrijske poravnave slik, je slednjo smiselno izboljševati.
	}
	\subsubsection*{Postopki za avtomatsko poravnavo medicinskih slik}
	\par{
	  Slikovna poravnava je postopek, pri katerem se poišče prostorsko preslikavo domene ene od slik (\emph{gibljive slike}) na domeno druge slike (\emph{negibne slike}), tako da se med slikama vzpostavi smiselna anatomska ali funkcionalna korespondenca. Gibljiva in negibna slika sta podani s preslikavama $S\colon\Omega_S\rightarrow\mathbb{R}$ in $T\colon\Omega_T\rightarrow\mathbb{R}$, kjer sta $\Omega_S, \Omega_T\subset\mathbb{R}^3$ običajno kvadrasti domeni, vrednosti preslikav pa predstavljajo slikovne intenzitete. Cilj slikovne poravnave je poiskati zvezno preslikavo $W:\Omega_S\rightarrow\Omega_T$, ki kar se da dobro poravna sliki. Zaželjene lastnosti preslikave $W$ so injektivnost, obrnljivost, odvedljivost.
	}
	\par{
	  Slikovno poravnavo sestavljajo tri ključne komponente, to so deformacijski model, kriterijska funkcija in optimizacijska metoda. Deformacijski model je predpis, ki podaja množico vseh možnih preslikav, med katerimi algoritem potem skuša poiskati optimalno. Če se da to množico parametrizirati s kako odprto podmnožico $\Theta\subset\mathbb{R}^n$, potem rečemo, da je model \emph{$n$-parametričen} oz.~da ima \emph{$n$ prostostnih stopenj}. Kriterijska funkcija je funkcija $f\colon\Theta\rightarrow\mathbb{R}$, ki kvantitativno ovrednoti kakovost poravnave slik. Pričakuje se, da $f$ zavzame globalni minimum pri tistem parametru $\mathbf{\theta}\in\Theta$, ki v danem modelu ponuja najboljšo poravnavo slik. Optimizacijska metoda je računski postopek, ki skuša poiskati minimum kriterijske funkcije. Obsežen pregled postopkov za slikovno poravnavo medicinskih slik je podan v \cite{sotiras2013}.
	}
	\subsubsection*{Slikovno vodena radioterapija glave in vratu}
	\par{
	  Rak na bomočju glave in vratu predstavlja okoli $3,2\%$ vseh malignih obolenj \cite{torre2015}. Pri obsevanju glave in vratu je potrebno paziti na več kritičnih struktur, ki so vitalnega pomena za ohranjanje kvalitete nadaljnjega življenja bolnika. Prekomerno obsevanje žlez slinavk poslabša izločanje sline in vodi v kserostomijo (suha usta), kar nadalje vodi v oteženo žvečenje in požiranje, zobno gnilobo, boleče grlo, izgubo okusa. Parotidne in podčeljustne žleze slinavke proizvajajo večino vse sline, večino sluzi, ki je sestavni del sline, pa proizvajajo preostale manjše žleze slinavke v ustni votlini. Opravljenih je bilo več študij, ki preučujejo vpliv radioterapije na vitalnost žlez slinavk in možnosti za njihovo prizanašanje: \cite{bhide2009,beetz2012,marzi2012,lee2014,yuan2014,vanluijk2015,tuomikoski2015,eisbruch2009}. Pri prekomernem obsevanju jezika, ustnega nebesa ter mišic žrela in grla prihaja do disfagije (motnje požiranja) ter vnetja ustne votline.
	}
	  
	  
	  
	
	%-------------------------
	% NASLEDNJI RAZDELEK
	\subsubsection*{}
	\par{}
	%-------------------------
	% PRICAKOVANI IZVIRNI PRISPEVKI K ZNANOSTI
	\subsection{Pričakovani izvirni prispevki k znanosti}
	\par{\noindent
		Pričakovani izvirni prispevki k znanosti v predlagani doktorski disertaciji so naslednji:
		\begin{enumerate}
			% 1. IZVIRNI PRISPEVEK
			\item Vrednotenje in primerjava različnih algoritmov za poravnavo slik na bazi slik glave in vratu
			% 2. IZVIRNI PRISPEVEK
			\item Metoda za razgradnjo večmodalnih slik
			% 3. IZVIRNI PRISPEVEK
			\item Izmisli si še kaj.

		\end{enumerate}
	}
	%-------------------------
	% METODOLOGIJA
	\subsection{Metodologija}
	\par{\noindent
		V skladu s predlaganimi pričakovanimi izvirnimi prispevki k znanosti je v nadaljevanju podana ustrezna metodologija:
		\begin{enumerate}
			% 1. IZVIRNI PRISPEVEK
			\item En odstavek.
			% 2. IZVIRNI PRISPEVEK
			\item En odstavek.
			% 3. IZVIRNI PRISPEVEK
			\item En odstavek.
			
		\end{enumerate}
	}
	%-------------------------
	\renewcommand{\bibsection}{\subsection{Izbrana literatura}}
	\bibliographystyle{./stil_literature.bst} % select author-year bibliography style in Slovenian language
	\bibliography{./literatura.bib}
	\newpage
	%-------------------------
	% ŽIVLJENJEPIS
	\newpage
	\section{Življenjepis}
	
	% Osebni podatki
	\subsection*{Osebni podatki}
	\begin{itemize}[align=left, itemsep=-0.1cm, leftmargin=4.25cm, labelwidth=*]
		\item[Ime in priimek:] Domen Močnik
		\item[Datum in kraj rojstva:] 24.11.1986, Novo mesto
		\item[Stalno bivališče:] Stopiče 73, 8322 Stopiče
		\item[Državljanstvo:] slovensko
	\end{itemize}
	
	% Kratek življenjepis
	\subsection*{Kratek življenjepis}
	\par{
		Domen Močnik se je rodil 24. novembra 1986 v Novem mestu. Po opravljeni poklicni maturi na Ekonomski šoli Novo mesto se je leta 2005 vpisal na visokošolski študijski program Praktična matematika na Fakulteti za matematiko in fiziko, Univerza v Ljubljani, na katerem je leta 2010 diplomiral z zagovorom diplomske naloge \textit{Večplastni piezoelektrični nosilec}. Študij je na isti fakulteti nadaljeval z vpisom v 3. letnik prvostopenjskega univerzitetnega študija Matematika, po zaključku pa se je leta 2012 na isti fakulteti vpisal na II. bolonjsko stopnjo programa Matematika. Septembra 2015 je zaključil študij z zagovorom magistrske naloge \textit{Hamiltonov princip v mehaniki kontinuuma}. Septembra 2015 se je vpisal na doktorski študij elektrotehnike na Fakulteti za elektrotehniko, Univerza v Ljubljani, ter se zaposlil kot mladi raziskovalec v Laboratoriju za slikovne tehnologije, kjer trenutno raziskuje na področju avtomatske računalniško podprte obdelave in analize medicinskih slik.
	}
	
	% Izobrazba
	\subsection*{Izobrazba}
	\begin{itemize}[align=left, itemsep=-0.1cm, leftmargin=1cm, labelwidth=*]
		\item[2005] Opravil poklicno maturo na Ekonomski šoli Novo mesto \\ Naziv: ekonomski tehnik
		\item[2010] Diplomiral na visokošolskem programu Praktična matematika na Fakulteti za matematiko in fiziko, Univerza v Ljubljani \\ Naziv: diplomirani inženir matematike
		\item[2012] Diplomiral na univerzitetnem programu Matematika na Fakulteti za matematiko in fiziko, Univerza v Ljubljani \\ Naziv: diplomirani matematik
		\item[2015] Magistriral na Fakulteti za matematiko in fiziko, Univerza v Ljubljani \\ Naziv: magister matematike
		\item[2015] Vpis na doktorski študij na Fakulteti za elektrotehniko, Univerza v Ljubljani \\ študijska smer: elektrotehnika
	\end{itemize}
	
	% Programi
	\subsection*{Programi}
	\begin{itemize}[align=left, itemsep=-0.1cm, leftmargin=2.5cm, labelwidth=*]
		\item[2015~-~\textit{danes}] P2-0232: Funkcije in tehnologije kompleksnih sistemov (Fakulteta za elektrotehniko, Univerza v Ljubljani)
	\end{itemize}
	

	%-------------------------
	
	
	%-------------------------
	% BIBLIOGRAFIJA
	\newpage
	\section{Bibliografija}
	\par{\noindent
		\textbf{Osebna bibliografija za obdobje 2010--2017\footnote{vir bibliografskih zapisov: vzajemna baza podatkov COBISS.SI/COBIB.SI (\href{http://www.cobiss.si}{http://www.cobiss.si}) z dne 25.5.2016}: \\}
		\par{\noindent \textbf{Monografije in druga zaključena dela} \\}
		\vspace{-0.3cm}
		\par{\noindent \textbf{[2.9] Magistrsko delo}}
		\begin{itemize}[align=right, itemsep=-0.05cm]
			\item[1.] MOČNIK, Domen. Hamiltonov princip v mehaniki kontinuuma : magistrsko delo. Ljubljana, 2015. [COBISS.SI-ID 17443673]
		\end{itemize}
		\par{\noindent \textbf{[2.11] Diplomsko delo}}
		\begin{itemize}[align=right, itemsep=-0.05cm]
			\item[2.] MOČNIK, Domen. Večplastni piezoelektrični nosilec : diplomska naloga. Ljubljana, 2010. [COBISS.SI-ID 15732057]
			\item[3.] MOČNIK, Domen. Zgornja polravnina kot model za hiperbolično geometrijo : delo diplomskega seminarja. Ljubljana, 2012. [COBISS.SI-ID 16659289]
		\end{itemize}

		%-------------------------
		\newpage
		\section{Izjava o sekundarnih podatkih}
		\par{		
		Prijavitelj teme doktorske disertacije Domen Močnik ter predvideni mentor doktorske disertacije izr.\ prof.\ dr.\ Tomaž Vrtovec izjavljava, da raziskave v povezavi z izdelavo doktorske disertacije z naslovom ``Slikovno vodena radioterapija glave in vratu'' temeljijo na sekundarnih podatkih in da so bili ti podatki v okviru raziskav anonimizirani.
		}
		\\\\\\
		\noindent\makebox[3cm][l]{Datum: 21.3.2017}
		\\\\\\
		\noindent\makebox[3cm][l]{Prijavitelj teme:} \hfill\makebox[5cm][l]{Predvideni mentor:}
		\\\\\\
		\noindent\makebox[3cm][l]{Domen Močnik, mag.\ mat.} \hfill\makebox[5cm][l]{izr.\ prof.\ dr.\ Tomaž Vrtovec}\\
		
		%-------------------------
		% DIPLOMA
		\newpage
		\section{Potrdilo o končanem izobraževanju}
		\par{ \noindent
			~\\
			\begin{center}
				\fbox{\includegraphics[angle=90,origin=c,width=14.25cm]{./fotokopija_magisterija}}
			\end{center}
		}
		%-------------------------
		
		%-------------------------
		% END DOCUMENT
	\end{document}
	%-------------------------
	